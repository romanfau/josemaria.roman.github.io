% jmroman_cv.tex, latex file
\newcommand{\version}{Madrid, 20 August 2021.}

\documentclass{article}
\usepackage{multirow}
\textwidth 6.5in  %165mm
%\textheight 8.9in %226mm-letter
\textheight 9.5in %241mm-a4paper
\topmargin -0.4in %10mm
\oddsidemargin 0pt
\evensidemargin 0pt
\parindent 0pt

%\pagestyle{empty}
\pagestyle{plain}

\begin{document}
%%%%%%%%%%%%%%%%%%%%%%%%%%%%%%%%%%%%%%%%%%%%%%%%%%%%%%%%%%%%%%
%%%%%%%%%%%     PERSONAL DATA    %%%%%%%%%%%%%%%%%%%%%%%%%%%%%
%%%%%%%%%%%%%%%%%%%%%%%%%%%%%%%%%%%%%%%%%%%%%%%%%%%%%%%%%%%%%%

{\huge \bf  Jos\'e Mar\'{\i}a Rom\'an Fa\'undez}

\bigskip \medskip

\begin{tabular}{lrl}
{\bf SP Control Technologies (frenetic.ai)} & & {\medskip} \\
{\small C/ Santa Engracia 108, 3$^o$ ext.-dcha.}
   & {\small {\bf Tel:}} & {\small +34\,91\,529\,6007} \\
{\small E-28003 Madrid, Spain}
%   & {\small {\bf Fax:}} & {\small +34\,91\,848\,7915} \\
%{\small}
   & {\small {\bf e-mail:}} & {\small josemaria.roman@frenetic.ai} \\
{\small}
   & {\small {\bf webpage:}} & {\small http://spcontroltechnologies.com/} \\
{\small}
   & {\small} & {\small https://www.frenetic.ai/}

%{\small C/ Gudari 20, 4 dcha-C}
%   & {\small {\bf Tel:}} & {\small +34\,637\,757\,325} \\
%   & {\small {\bf e-mail:}} & {\small josemaria.roman@alumni.uam.es} \\
%{\small E-48340 AMOREBIETA (Vizcaya) Spain}
%   & {\small {\bf e-mail:}} & {\small josemaria.roman@alumni.uam.es} \\
%{\small ~}
 %  & {\small {\bf webpage:}} & {\small http://romanfau.github.io/}
\end{tabular}

\bigskip

\hrule

%%%%%%%%%%%%%%%%%%%%%%%%%%%%%%%%%%%%%%%%%%%%%%%%%%%%%%%%%%%%%%
%\begin{tabular}{lrl}
%\multirow{4}{102mm}{\huge \bf  Jos\'e Mar\'{\i}a Rom\'an Fa\'undez}
%& {\small {\bf Date of birth:}} & {\small March 18th, 1971} \\
%& {\small {\bf Place of birth:}} & {\small Zamora, Spain} \\
%& {\small {\bf Nationality:}} & {\small Spanish} \\
%& {\small {\bf I.D., Passport:}} & {\small 11\,894\,268\,-\,W} 
%\end{tabular}

%\begin{tabular}{lrl}
%{\small C/ Gudari 20, 4 dcha.-C}
%   & {\small {\bf Tel:}} & {\small +34\,94\,673\,3814} \\
%{\small E-48340 AMOREBIETA}
%   & {\small {\bf e-mail:}} & {\small josemaria.roman@alumni.uam.es} \\
%{\small (Vizcaya) SPAIN}
%   & {\small {\bf webpage:}} & {\small http://romanfau.gitnub.io/}
%\end{tabular}

%\bigskip

%\hrule

%%%%%%%%%%%%%%%%%%%%%%%%%%%%%%%%%%%%%%%%%%%%%%%%%%%%%%%%%%%%%%
%%%%%%%%%%%%    SUMMARY     %%%%%%%%%%%%%%%%%%%%%%%%%%%%%%%%%%
%%%%%%%%%%%%%%%%%%%%%%%%%%%%%%%%%%%%%%%%%%%%%%%%%%%%%%%%%%%%%%

\section*{Summary}

COO at SP Control Technologies (frenetic.ai). Laboratory and O{\&}M Services Director in the R{\&}D, Aftersales Service Center at Yingli Green Energy Europe for nine years. Project Researcher at Centro L\'aser UPM. Director of Photovoltaic Systems Quality and Control Laboratory for five years at INGENIA Solar Energy and CENER. Technical Manager of the glass-glass PV modules production line for BIPV. Implementation of the quality management systems ISO 17025 and ISO 9001. Ph.D.\ in Physics with two years research experience in the USA. Publications in international journals with referee. Five years living abroad. Fluent English.

\bigskip

\hrule

%%%%%%%%%%%%%%%%%%%%%%%%%%%%%%%%%%%%%%%%%%%%%%%%%%%%%%%%%%%%%%
%%%%%%%%%%%%    CURRENT SITUATION     %%%%%%%%%%%%%%%%%%%%%%%%
%%%%%%%%%%%%%%%%%%%%%%%%%%%%%%%%%%%%%%%%%%%%%%%%%%%%%%%%%%%%%%

\section*{Current Situation}

{\bf COO}

\medskip
Dec.~2020-Present.
{\bf SP Control Technologies (frenetic.ai)},
Madrid, Spain.

\begin{itemize}\itemsep 0pt
\item Management and general organization of the company: project management, purchasing process and structure of Human Resources.
\item Compliance of the design and programming works, as well as the laboratory tests with quality management norm ISO 9001.
\item Follow-up and compilation of the annual justification report for the European Project 953971-FRENETIC.
\item Optimization of physical models applied to magnetic components for electrical and electronic equipment using Artificial Intelligence.
\end{itemize}

%%%%%%%%%%%%%%%%%%%%%%%%%%%%%%%%%%%%%%%%%%%%%%%%%%%%%%%%%%%%%%
%%%%%%%%%%%%    PROFESSIONAL EXPERIENCE     %%%%%%%%%%%%%%%%%%
%%%%%%%%%%%%%%%%%%%%%%%%%%%%%%%%%%%%%%%%%%%%%%%%%%%%%%%%%%%%%%

\section*{Professional Experience}

{\bf Laboratory and O{\&}M Director in the R{\&}D, Aftersales Service Center}

\medskip
Feb.~2012-Nov.~2020.
{\bf Yingli Green Energy Europe},
Madrid, Spain.

\begin{itemize}\itemsep 0pt
\item O{\&}M Services supervision, performance reports generation.
\item Training courses for O{\&}M testing of PV plants.
\item Monitoring of BIPV system integrated in YGEE building.
\item Installation of an off-grid PV system, and planned project in energy demand management.
\item Aftersales and R{\&}D project management.
\item Product analysis and test in the laboratory and in the field.
\item New tests definition and compliance with the guidelines of ISO 17025.
\item Member of AENOR SC-82 for PV modules and systems normalization.
\item PV plants technical due-diligence.
\item PV module manufacturing facilities audits.
\end{itemize}

\newpage

{\bf Project Researcher}

\medskip
Feb.~2011-Jan.~2012.
{\bf Centro L\'aser UPM},
Madrid, Spain.

\begin{itemize}\itemsep 0pt
\item Researched on deposition of metallic contacts on PV solar cells using laser.
\item Operation of q-switched and cw-lasers, Confocal microscope, SEM and EDX spectroscope.
\end{itemize}

{\bf Director of the Photovoltaic Systems Quality and Control Laboratory}

\medskip
May~2007-Jan.~2011.
{\bf INGENIA Solar Energy},
Albacete, Spain.

\begin{itemize}\itemsep 0pt
\item Quality control measurement of PV modules under international norms.
\item Inspections and Performance measurements of PV plants.
\item Design and dimensioning of PV plants and production estimates calculations.
\item Implementation of the laboratory quality management system ISO 17025.
\end{itemize}

{\bf Manager of the Photovoltaic Systems Service}

\medskip
Apr.~2006-Apr.~2007.
{\bf CENER - Spanish Center for Renewable Energy},
Navarra, Spain.

\begin{itemize}\itemsep 0pt
\item Manager of the PV Modules Testing Laboratory.
\item Manager of the PV Installation Design group.
\item Implementation of the management system of the Service.
\item Management of commercial and development projects.
\end{itemize}

%\newpage

%%%%%%%%%%%%%%%%%%%%%%%%%%%%%%%%%%%%%%%%%%%%%%%%%%%%%%%%%%%%%%
{\bf Technical Manager of the PV module production line}

\medskip
Oct.~2004-Mar.~2006. 
{\bf Romag Ltd.} 
Consett, Co. Durham, UK.

\begin{itemize}\itemsep 0pt
\item Design of the lamination process for custom made glass-glass solar modules and standard glass-Tedlar modules.
\item Definition of the production process for custom-made glass-glass solar modules, as well as for glass-Tedlar modules for qualification under IEC 61215 standard.
\item Compliance of the production process with quality management norm ISO 9001.
\item Training of personnel in the production process and operation of vacuum laminator and flash-tester equipment
\item Dealing with suppliers and customers.
\end{itemize}

%%%%%%%%%%%%%%%%%%%%%%%%%%%%%%%%%%%%%%%%%%%%%%%%%%%%%%%%%%%%%%
%%%%%%%%%%%%%   EDUCATION   %%%%%%%%%%%%%%%%%%%%%%%%%%%%%%%%%%
%%%%%%%%%%%%%%%%%%%%%%%%%%%%%%%%%%%%%%%%%%%%%%%%%%%%%%%%%%%%%%

\section*{Education}

{\bf Senior Management Course (CADE)}

\medskip
Nov.~2009-Jun.~2010. {\bf FEDA-Fundesem}. Albacete, Spain.

\begin{itemize}\itemsep 0pt
\item {\bf Objective:} To develop knowledge and basic skills for the overall management of a company.
\begin{itemize} 
\item Strategic management:  Strategic management and Economic environment.
\item Marketing and commercialization:  Marketing planning and Sales planning.
\item Economic and Financial:  Financial analysis and Management control.
\item Human Resources and Management skills:  Communication and  People management.
\end{itemize}
\end{itemize}

\newpage

%%%%%%%%%%%%%%%%%%%%%%%%%%%%%%%%%%%%%%%%%%%%%%%%%%%%%%%%%%%%%%
\medskip
{\bf EUREC Agency European Master in Renewable Energy}

\medskip
Oct.~2003-Sept.~2004. {\bf Universidad de Zaragoza}. Zaragoza, Spain.

\begin{itemize}\itemsep 0pt
\item {\bf Introduction:} CIRCE, University of Zaragoza, Spain.
\begin{itemize}
\item Introduction to the technical and socio-economic aspects of the Renewable Energies: Wind Power, Photovoltaic Solar, Thermal Solar, Hydroelectric Power, and Biomass.
\end{itemize}
\item {\bf Specialization:} {\bf Photovoltaic} Solar Energy, Northumbria University, Newcastle, UK.
\begin{itemize}
\item Physics, design and technology of solar cells and modules, and design of Photovoltaic systems.
\end{itemize}
\item {\bf Final Project:} Architectural PV solar modules production line: Process description and implementation.
\begin{itemize}
\item Jun.~2004-Sept.~2004. Romag Ltd., Consett, Co. Durham, UK.
\item Design of the lamination process for custom made glass-glass solar modules.
\end{itemize}
\end{itemize}

%\newpage

%%%%%%%%%%%%%%%%%%%%%%%%%%%%%%%%%%%%%%%%%%%%%%%%%%%%%%%%%%%%%%

\medskip
{\bf Ph.D.\ in Physics}

\medskip
Oct.~1994-Oct.~1998. {\bf Universitat de Barcelona}. Barcelona, Spain.

\begin{itemize}\itemsep 0pt
\item Training in Quantum Field Theory, General Relativity, Nuclear Physics, 
Stochastic equations and noise, and String Theory.
\item {\bf Thesis dissertation:} Low Energy Properties of Magnetic Systems.
\begin{itemize}
\item {\bf Supervisor:} Joan Soto Riera.
\item Combined crystallography, optics and effective field theory to determine the dynamics of spinwave excitations and non-reciprocal effects in ferromagnets and antiferromagnets.
\item Applied a recursive numerical technique to compute ground state properties of quantum spin ladders.
\end{itemize}
\end{itemize}

%%%%%%%%%%%%%%%%%%%%%%%%%%%%%%%%%%%%%%%%%%%%%%%%%%%%%%%%%%%%%%
\medskip
{\bf Bachelor of Science in Physics, specialized in Solid State Physics}
%GPA 8.6/10.0.

\medskip
Sept.~1989-June~1994.
{\bf Univ.\ del Pa\'{\i}s Vasco - Euskal Herriko Unibertsitatea}.
Leioa, Spain.

\begin{itemize}\itemsep 0pt
\item Training in Crystallography, Lattice dynamics and phonons, Electronic structure and semiconductors, Quantum Mechanics, Optics, Dielectric materials, together with several sets of related experiments.
\end{itemize}

%%%%%%%%%%%%%%%%%%%%%%%%%%%%%%%%%%%%%%%%%%%%%%%%%%%%%%%%%%%%%%
%%%%%%%%%%     RESEARCH EXPERIENCE     %%%%%%%%%%%%%%%%%%%%%%%
%%%%%%%%%%%%%%%%%%%%%%%%%%%%%%%%%%%%%%%%%%%%%%%%%%%%%%%%%%%%%%

\section*{Research Experience}

Jul.~2001-Nov.~2003. 
{\bf Intituto de F\'{\i}sica Te\'orica, CSIC-UAM}. 
Madrid, Spain. 

\begin{itemize}\itemsep 0pt
\item Studied the cyclic flow of the RG and its effects in 
superconductors and spin chains.

\item Obtained the main features of the ground state and excitations in small 
grains of materials with superconducting correlations.
\end{itemize}

%\newpage

%%%%%%%%%%%%%%%%%%%%%%%%%%%%%%%%%%%%%%%%%%%%%%%%%%%%%%%%%%%%%%
Jan.~2001-Jul.~2001. {\bf Universidade de \'Evora}. \'Evora, Portugal.

\begin{itemize}\itemsep 0pt
\item Collaborated with ISTAS in the preparation of a program of scientific 
conferences.

\item Described the finite energy 1-D Hubbard model excitations as
combinations of spinons and holons.
\end{itemize}

%%%%%%%%%%%%%%%%%%%%%%%%%%%%%%%%%%%%%%%%%%%%%%%%%%%%%%%%%%%%%%
Nov.~1998-Dec.~2000. 
{\bf University of Illinois at Urbana-Champaign}. Urbana, Illinois, USA.

\begin{itemize}\itemsep 0pt
\item Studied the suppression of the superconducting order parameter
around magnetic impurities in $d$-wave superconductors. 

\item Extended the effective theory of spin waves to canted phases
and applied it to the study of magnetic excitations in doped manganites.
\end{itemize}

%%%%%%%%%%%%%%%%%%%%%%%%%%%%%%%%%%%%%%%%%%%%%%%%%%%%%%%%%%%%%%
Oct.~1994-Oct.~1998. {\bf Universitat de Barcelona}. Barcelona, Spain.

\begin{itemize}\itemsep 0pt
\item Defined a continuum model to obtain the ground state phase diagram 
of doped manganites as a function of doping.

\item Combined crystallography, optics and effective field theory to 
determine the dynamics of spin wave excitations and non-reciprocal 
effects in ferromagnets and antiferromagnets.
\end{itemize}

%%%%%%%%%%%%%%%%%%%%%%%%%%%%%%%%%%%%%%%%%%%%%%%%%%%%%%%%%%%%%%
Oct.~1997-Dec.~1997. {\bf Consejo Superior de Investigaciones 
Cient\'{\i}ficas (CSIC)}. Madrid, Spain. 

\begin{itemize}\itemsep 0pt
\item Applied a recursive numerical technique to compute 
ground state properties of quantum spin ladders.
\end{itemize}

%\newpage

%%%%%%%%%%%    PUBLICATIONS     %%%%%%%%%%%%%%%%%%%%%%%%%%%%%%%

%\subsubsection*{Publications}
\medskip
{\bf Publications:}
%
%\begin{itemize}\itemsep 0pt
%\item
Fourteen publications in international journals with referee.
%\end{itemize}

%%%%%%%%%    SEMINARS AND COMMUNICATIONS    %%%%%%%%%%%%%%%%%%%%%%%

%\subsubsection*{Seminars and Communications in Congresses}
\medskip
{\bf Seminars and Communications in Congresses:}
%
%\begin{itemize}\itemsep 0pt
%\item 
Seventeen oral and two poster invited presentations 
in several universities and international conferences.
%\end{itemize}

%%%%%%%%%    COURSES AND CONFERENCES    %%%%%%%%%%%%%%%%%%%%%%%%%%

%\subsection*{Courses and Conferences}
\medskip
{\bf Courses and Conferences:}
%
%\begin{itemize}\itemsep 0pt
%\item 
Twenty two international courses and conferences attended.
%\end{itemize}

%%%%%%%%%    PROJECTS SUPERVISED    %%%%%%%%%%%%%%%%%%%%%%%%%%

%\subsubsection*{Projects Supervised}
\medskip
{\bf Projects Supervised}
%
%\begin{itemize}\itemsep 0pt
%\item 
Eight research projects supervised in Photovoltaic Solar Energy.
%\end{itemize}

%%%%%%%%%    OUTREACH CONTRIBUTIONS %%%%%%%%%%%%%%%%%%%%%%%%%%

%\subsubsection*{Outreach Contributions}
\medskip
{\bf Outreach Contributions}
%
%\begin{itemize}\itemsep 0pt
%\item 
One presentation and one article published.
%\end{itemize}

%%%%%%%%%%    TEACHING EXPERIENCE     %%%%%%%%%%%%%%%%%%%%%%%

\subsubsection*{Teaching Experience}

\begin{itemize}\itemsep 0pt
\item 1998. 
Taught problems of the Eighth Semester subject
Nuclear Physics and Particles. 
Facultat de F\'{\i}sica. 
Universitat de Barcelona. 
Barcelona, Spain.
%30 hours.

\item 1995. 
Taught problems of the Second Semester subject 
Classical Mechanics and Waves. 
Facultat de F\'{\i}sica. 
Universitat de Barcelona. 
Barcelona, Spain.
%45 hours.
\end{itemize}

%%%%%%%%%%%%%%    FELLOWSHIPS    %%%%%%%%%%%%%%%%%%%%%%%%%%%%%

\subsubsection*{Fellowships}

\begin{itemize}\itemsep 0pt
\item Jan.~2001-Jul.~2001.
Postdoctoral fellowship from the Foundation for Science and
Technology of the Portuguese Government.

\item Nov.~1998-Sept.~2000.
Postdoctoral FPI fellowship from the Dpt.\ of Education,
Universities and Research of the Basque Country Government.

\item Oct.~1994-Sept.~1998.
Predoctoral FPI fellowship from the Dpt.\ of Education,
Universities and Research of the Basque Country Government.
\end{itemize}

%\newpage

%%%%%%%%%%%%%%%%%%%%%%%%%%%%%%%%%%%%%%%%%%%%%%%%%%%%%%%%%%%%%%
%%%%%%%%%%%%     PROFESSIONAL SKILLS      %%%%%%%%%%%%%%%%%%%%
%%%%%%%%%%%%%%%%%%%%%%%%%%%%%%%%%%%%%%%%%%%%%%%%%%%%%%%%%%%%%%

\section*{Professional Skills}

\begin{itemize}\itemsep 0pt
\item Excellent organizational skills for the business administration complying with the quality norms ISO 9001 and ISO 17025.

\item Extensive experience in solving complex mathematical and physical 
problems through a combination of analytical and numerical techniques.

\item Experienced in multidisciplinary work and quick adaptability 
to new topics. Quick learning ability and teaching ability.
\end{itemize}

%%%%%%%%%%%%%%%%%%%%%%%%%%%%%%%%%%%%%%%%%%%%%%%%%%%%%%%%%%%%%%
%%%%%%%%%%%%%%%    COMPUTER SKILLS      %%%%%%%%%%%%%%%%%%%%%%
%%%%%%%%%%%%%%%%%%%%%%%%%%%%%%%%%%%%%%%%%%%%%%%%%%%%%%%%%%%%%%

\section*{Computer Skills}

\begin{itemize}\itemsep 0pt
\item Expert in office and calculation tools: MS Word, Excel and Access, \TeX and \LaTeX.

\item Projects and tests laboratory management system based on Access and Excel.

\item Knowledge of Auto-CAD, HTML (http://dftuz.unizar.es/ftzar/mapoftheweb.html), Java, JSON and XML.

\item Strong experience programming in C++, FORTRAN, Python, VBA, Mathematica and Octave for the solution of problems numerically.

\item Combined C++ code with FORTRAN and XmGrace libraries to obtain real-time graphical outputs.

\item Monitoring application by reading and analyzing data with VBA and Excel from Access database.

\item Monitoring application by reading and analyzing data with Python and MS SQLServer.

\item Linux administration at intermediate level.  
User of UNIX (AIX, IRIX), Windows and Windows NT platforms.
\end{itemize}

%%%%%%%%%%%%%%%%%%%%%%%%%%%%%%%%%%%%%%%%%%%%%%%%%%%%%%%%%%%%%%
%%%%%%%%%%%%%%     LANGUAGES     %%%%%%%%%%%%%%%%%%%%%%%%%%%%%
%%%%%%%%%%%%%%%%%%%%%%%%%%%%%%%%%%%%%%%%%%%%%%%%%%%%%%%%%%%%%%

\section*{Languages}
\begin{itemize}\itemsep 0pt
\item Spanish, English, and moderate knowledge of Portuguese, 
Catalan and Basque.
\end{itemize}

%%%%%%%%%%%%%%%%%%%%%%%%%%%%%%%%%%%%%%%%%%%%%%%%%%%%%%%%%%%%%%%

\bigskip

\hrule

\newpage

%%%%%%%%%%%%%%%%%%%%%%%%%%%%%%%%%%%%%%%%%%%%%%%%%%%%%%%%%%%%%%%
%%%%%%%%%%%    PUBLICATIONS     %%%%%%%%%%%%%%%%%%%%%%%%%%%%%%%
%%%%%%%%%%%%%%%%%%%%%%%%%%%%%%%%%%%%%%%%%%%%%%%%%%%%%%%%%%%%%%%

\section*{Publications}

\begin{enumerate}

\item A. LeClair, J. M. Rom\'an and G. Sierra,
{\it Log-periodic Behavior of Finite-size Effects in Field Theories with RG Limit Cycles}.
{\it Nucl.~Phys.} {\bf B700} (2004) 407-435.
%(hep-th/0312141).
%November 15, 2004.

\item G. Sierra, J. M. Rom\'an and J. Dukelsky, 
{\it The Elementary Excitations of the BCS Model in the Canonical Ensemble}.
{\it Int.~J.~Mod.~Phys.} {\bf A19S2} (2004) 381-395.
%(cond-mat/0301417).
%May, 2004.

\item J. M. P. Carmelo, J. M. Rom\'{a}n and K.Penc,
{\it Charge and Spin Quantum Fluids Generated by Many-Electron Interactions}.
{\it Nucl.~Phys.} {\bf B683} (2004) 387-422.
%(cond-mat/0302044).
%March 12, 2004.

\item A. LeClair, J. M. Rom\'{a}n and G. Sierra, 
{\it Russian Doll Renormalization Group and Superconductivity}. 
{\it Phys.~Rev.} {\bf B69} (2004) 020505 (4 pages).
%(cond-mat/0211338).
%January 27, 2004.

\item A. LeClair, J. M. Rom\'{a}n and G. Sierra, 
{\it Russian Doll Renormalization Group, Kosterlitz-Thouless Flows, 
and the Cyclic sine-Gordon Model}. 
{\it Nucl.~Phys.} {\bf B675} (2003) 584-606.
%(hep-th/0301042).
%December 29, 2003.

\item J. Dukelsky, J. M. Rom\'an and G. Sierra,
Comment on {\it Polynomial-time Simulation of Pairing Models 
on a Quantum Computer}.
{\it Phys.~Rev.~Lett.} {\bf 90} (2003) 249803.
%(quant-ph/0305139).
%June 20, 2003

\item J. M. Rom\'an, G. Sierra and J. Dukelsky, 
{\it Elementary Excitations of the BCS Model in the Canonical Ensemble},
{\it Phys.~Rev.} {\bf B67} (2003) 064510 (6 pages).
%(cond-mat/0207640).
%February 28, 2003.

\item J. M. Rom\'an, G. Sierra and J. Dukelsky, 
{\it Large $N$ Limit of the Exactly Solvable BCS Model: 
Analytics versus Numerics},
{\it Nucl.~Phys.} {\bf B634} (2002) 483-510.
%(cond-mat/0202070).
%July 15, 2002.

\item J. M. Rom\'an and J. Soto, 
{\it Spin Waves in Canted Phases: An Application to Doped Manganites}, 
{\it Phys.~Rev.} {\bf B62} (2000) 3300-3315.
%(cond-mat/9911471).
%August 1, 2000.

\item J. M. Rom\'an and J. Soto, 
{\it Continuum Double Exchange Model},
{\it Phys.~Rev.} {\bf B59} (1999) 11418-11423.
%(cond-mat/9810389).
%May 1, 1999.

\item J. M. Rom\'an and J. Soto, 
{\it Spin Wave Mediated Non-Reciprocal Effects in Antiferromagnets}, 
{\it Ann.~Phys.} {\bf 273} (1999) 37-57.
%(cond-mat/9709299).
%April 10, 1999.

\item J. M. Rom\'an and J. Soto, 
{\it Effective Field Theory Approach to Ferromagnets and Antiferromagnets 
in Crystalline Solids}, 
{\it Int.~J.~Mod.~Phys.} {\bf B13} (1999) 755-789.
%(cond-mat/9709298).
%March 20, 1999.

\item J. M. Rom\'an, G. Sierra, J. Dukelsky and M. A. Mart\'{\i}n-Delgado, 
{\it The Matrix Product Approach to Quantum Spin Ladders}, 
{\it J.~Phys.} {\bf A31} (1998) 9729-9759.
%(cond-mat/9802150).
%December 4, 1998.

\item J. M. Rom\'an and R. Tarrach, 
{\it The Regulated Four-Parameter One-Dimensional Point Interaction}, 
{\it J.~Phys.} {\bf A29} (1996) 6073-6085.
%(hep-th/9511010).
%September 21, 1996.
\end{enumerate}

%%%%%%%%%%%    OUTREACH PUBLICATIONS     %%%%%%%%%%%%%%%%%%%%%%

\subsection*{Outreach Publications}

\begin{enumerate}
\item J. M. Rom\'an,
{\it Rendimiento de instalaciones fotovoltaicas conectadas a la red el\'ectrica y su aplicaci\'on a BIPV}.
{\it cicNetwork Ciencia y Tecnolog�a} {\bf n$^o$ 11} (mayo 2012) 36-42.
%(11-0002_20120517_pub)).
%May 17, 2012.
\end{enumerate}

%\section*{Preprints}

\newpage

%%%%%%%%%%%%%%%%%%%%%%%%%%%%%%%%%%%%%%%%%%%%%%%%%%%%%%%%%%%%%%
%%%%%%%%%%%%%%%%%%%%    SEMINARS    %%%%%%%%%%%%%%%%%%%%%%%%%%
%%%%%%%%%%%%%%%%%%%%%%%%%%%%%%%%%%%%%%%%%%%%%%%%%%%%%%%%%%%%%%

\section*{Seminars}

\begin{enumerate}
\item
{\it Yingli Spain recommendations for O\&M tasks in PV plants in desertic areas of Per\'u and Senegal}. 
Workshop {\it INVIVONexth on O\&M and Acceptance for PV plants in arid climates}.
CIEMAT, Madrid, Spain.
Dec.~14, 2017.

\item
{\it Yingli Spain R\&D, Aftersales Service Center BIPV system: one year experience}. 
Workshop {\it Eco-design from PV to BIPV}.
CIEMAT, Madrid, Spain.
Mar.~14-17, 2017.

\item
{\it Estado de desarrollo actual de la tecnolog{\'\i}a fotovoltaica}. 
Jornada UPM-UNEF-AS: Cambio energ\'etico y autoconsumo solar en Espa\~na - los retos para la nueva legislatura.
ETS de Ingenier{\'\i}a y Dise\~no Industrial (UPM), Madrid. Spain.
Dec.~16, 2015.

\item
{\it El Futuro de la Energ{\'\i}a Solar Fotovoltaica}.
Club Espa\~nol de la Energ{\'\i}a.
Madrid, Spain.
May~11, 2006. 

\item
{\it Cyclic RG Theories and c-Function Behavior in Finite Size Systems}.
Dpt.~d'Estruc\-tura i Constituents de la Mat\`eria.
Facultat de F\'{\i}sica. Universitat de Barcelona.
Barcelona, Spain.
Jan.~8, 2004.

\item
{\it Limit Cycles in the Renormalization Group of the BCS and the sine-Gordon models}. 
Dpto.\ de F\'{\i}sica Te\'orica. Universidad de Zaragoza.
Zaragoza, Spain.
Oct.~29, 2003.

\item
{\it Russian Doll Renormalization Group and Superconductivity}. 
Instituto de Ciencias de Materiales de Madrid, CSIC. 
Madrid, Spain.
Mar.~20, 2003.

\item
{\it Superconductivity in Small Grains: Ground State Thermodynamic 
Limit and Excitations}.
Dpto.~de F\'{\i}sica Aplicada. Universidad de Alicante.
Alicante, Spain.
Nov.~21, 2002.

\item
{\it Superconductivity in Small Grains: Ground State Thermodynamic 
Limit and Excitations}.
Dpt.~d'Estruc\-tura i Constituents de la Mat\`eria.
Facultat de F\'{\i}sica. Universitat de Barcelona.
Barcelona, Spain.
Jul.~4, 2002.

\item 
{\it Disentangling Canted Phases and Phase Separation Regions 
in Doped Manganites with Spin Waves}.
Centro de F\'{\i}sica das Interac\c c\~oes Fundamentais. 
Instituto Superior T\'ecnico.
Lisbon, Portugal.
Apr.~3, 2001.

\item 
{\it Disentangling Canted Phases and Phase Separation Regions 
in Doped Manganites with Spin Waves}.
Dpto. de F\'{\i}sica. 
Faculdade de Ci\^encias.
Universidade do Porto.
Oporto, Portugal.
Jan.~26, 2001.

\item 
{\it Disentangling Canted Phases and Phase Separation Regions 
in Doped Manganites with Spin Waves}.
Instituto de Ciencias de Materiales de Madrid, CSIC. 
Madrid, Spain.
Jan.~10, 2000.

\item 
{\it Doped Manganites Phase Diagram from a Continuum Double Exchange Model}. 
Dpto.~de F\'{\i}sica del Estado S\'olido. Facultad de Qu{\'\i}mica.  
Universidad del Pa\'{\i}s Vasco - Euskal Herriko Unibertsitatea (UPV-EHU). 
San Sebastian-Donostia, Spain.
Sept.~9, 1999.

\item 
%{\it Teor\'{\i}as Efectivas Aplicadas a Cristales 
%Antiferromagn\'{e}ticos: Efectos no Rec\'{\i}procos Mediados 
%por Ondas de Spin}
{\it Effective Field Theory Approach to Spin Wave Mediated 
Non-Reciprocal Effects in Antiferromagnetic Crystals}. 
Instituto de Ciencias de Materiales de Madrid, CSIC. 
Madrid, Spain.
Dec.~10, 1997. 

\item 
{\it Effective Field Theory Approach to Spin Wave Mediated 
Non-Reciprocal Effects in Antiferromagnetic Crystals}. 
Dpto.\ de F\'{\i}sica Te\'orica e Historia de la Ciencia. 
Facultad de Ciencias. 
Universidad del Pa\'{\i}s Vasco - Euskal Herriko Unibertsitatea (UPV-EHU). 
Leioa, Spain.
Nov.~13, 1997. 
\end{enumerate}

%%%%%%%%%%%   COMMUNICATIONS IN CONGRESSES    %%%%%%%%%%%%%%%%

\subsection*{Communications in Congresses}

\begin{enumerate}
%\setcounter{enumi}{11}
\item
{\it Russian Doll Renormalization Group and Superconductivity}. 
Trobada de Nadal 2002 (Christmas Meeting 2002) del dpt.~d'Estructura 
i Constituents de la Mat\`eria. 
Barcelona, Spain.
Dec.~19-20, 2002.

\item
{\it Disentangling Canted Phases from Phase Separation Regions in Doped 
Manganites with Spin Waves}. 
Euroconference on {\it Transport and Dynamics in Complex Electronic Materials}.
Oporto, Portugal. 
Sept.~3-7, 2001. 

\item 
{\it Disentangling Canted Phases from Phase
Separation Regions in Doped Manganites with Spin Waves}.
2000 March Meeting of the American Physical Society. 
Minneapolis, MN, USA. 
Mar.~20-24, 2000.

\item 
{\it Doped Manganites Phase Diagram from a Continuum Double Exchange Model}. 
Summer School on {\it Exotic States in Quantum Nanostructures}. 
Windsor, UK. 
Aug.~16-29, 1999.
\end{enumerate}

%%%%%%%%%%%   OUTREACH PRESENTATIONS    %%%%%%%%%%%%%%%%%%%%%%%

\subsection*{Outreach Presentations}

\begin{enumerate}
%\setcounter{enumi}{11}
\item
{\it El bos\'on de Higgs: la simetr�a y la masa}. 
Ibarrangelu (Vizcaya), Spain.
Dec.~23, 2012. 
\end{enumerate}

\newpage

%%%%%%%%%%%%%%%%%%%%%%%%%%%%%%%%%%%%%%%%%%%%%%%%%%%%%%%%%%%%%%
%%%%%%%%%%%      COURSES AND CONFERENCES     %%%%%%%%%%%%%%%%%
%%%%%%%%%%%%%%%%%%%%%%%%%%%%%%%%%%%%%%%%%%%%%%%%%%%%%%%%%%%%%%

\section*{Courses and Conferences}

\begin{enumerate}
\item
34th European PV Solar Energy Conference EU PVSEC 2018.
Brussels, Belgium.
Sept.~24-28, 2018.

\item
Workshop {\it INVIVONexth on O\&M and Acceptance for PV plants in arid climates}.
CIEMAT, Madrid, Spain.
Dec.~14, 2017. 

\item
Principios de Almacenamiento de Energ{\'\i}a.
CIEMAT, Madrid, Spain.
Oct.~2-6, 2017. 

\item
33rd European PV Solar Energy Conference EU PVSEC 2017.
Amsterdam, The Netherlands.
Sept.~25-29, 2017.

\item
Workshop {\it Eco-design from PV to BIPV}.
CIEMAT, Madrid, Spain.
Mar.~14-17, 2017. 

\item
32nd European PV Solar Energy Conference EU PVSEC 2016.
Munich, Germany.
Jun.~20-24, 2016.

\item
Jornada UPM-UNEF-AS:  Cambio energ\'etico y autoconsumo solar en Espa\~na - los retos para la nueva legislatura.
ETS de Ingenier{\'\i}a y Dise\~no Industrial (UPM), Madrid, Spain.
Dec.~16, 2015. 

\item
31st European PV Solar Energy Conference EU PVSEC 2015.
Hamburg, Germany.
Sept.~14-18, 2015.

\item
29th European PV Solar Energy Conference EU PVSEC 2014.
Amsterdam, The Netherlands.
Sept.~22-26, 2014.

\item
28th European PV Solar Energy Conference EU PVSEC 2013.
Paris, France.
Sept.~30-Oct.~4, 2013.

\item
27th European PV Solar Energy Conference EU PVSEC 2012.
Frankfurt, Germany.
Sept.~24-28, 2012.

\item
21st European PV Solar Energy Conference EU PVSEC 2006.
Dresden, Germany.
Sept.~4-8, 2006.

\item
Trobada de Nadal 2002 (Christmas Meeting 2002) del dpt.~d'Estructura 
i Constituents de la Mat\`eria. 
Universitat de Barcelona.
Barcelona, Spain.
Dec.~19-20, 2002.

\item
Euroconference on 
{\it Transport and Dynamics in Complex Electronic Materials}.
Oporto, Portugal. 
Sept.~3-7, 2001. 

\item 
2000 March Meeting of the American Physical Society. 
Minneapolis, MN, USA. 
Mar.~20-24, 2000.

\item 
Summer School on 
{\it Exotic States in Quantum Nanostructures}.
Windsor, UK. 
Aug.~16-29, 1999.

\item 
Summer School on 
{\it Dynamic Correlations in Many Fermion Systems}. 
Vila Nova de Cerveira, Portugal. 
Jul.~14-25, 1997.

\item
Advanced School on {\it Non-Perturbative Quantum Field Theory}. 
Pe\~n\'{\i}scola, Spain. 
Jun.~2-6, 1997.

\item 
%{\it Jornadas sobre Teor\'{\i}a Cu\'antica de Campos en 
%Sistemas de Baja Dimensi\'on y Materia Condensada}. 
Conference on 
{\it Quantum Field Theory in Low Dimensional and Condensed Matter Systems}.
Instituto de Ciencias de Materiales, CSIC. Madrid, Spain. 
Nov.~7-8, 1996.

\item  
%IV Escuela de Oto\~no de F\'{\i}sica Te\'orica,
%{\it M\'etodos no Perturbativos en Teor\'{\i}a Cu\'antica de Campos}.
IV Autumn School of Theoretical Physics on
{\it Non-perturbative Methods in Quantum Field Theory}.
Santiago de Compostela, Spain. 
Sept.~2-14, 1996.

\item  
%Cursos de Verano de la Universidad Complutense, 
%{\it Strongly Correlated Magnetic and Superconducting Systems}.
Complutense University Summer School on
{\it Strongly Correlated Magnetic and Superconducting Systems}.
San Lorenzo del Escorial, Spain. 
Jul.~15-19, 1996.

\item  
%III Escuela de Oto\~no de F\'{\i}sica Te\'orica, 
%{\it Introducci\'on al Modelo Standard de las Interacciones Fundamentales}. 
III Autumn School of Theoretical Physics on
{\it Introduction to the Standard Model of the Fundamental Interactions}.
Santiago de Compostela, Spain. 
Sept.~4-16, 1995. 
\end{enumerate}

\newpage

%%%%%%%%%%%      OTHER COURSES     %%%%%%%%%%%%%%%%%%%%%%%%%%%

\subsection*{Other Courses}

\begin{enumerate}
%\setcounter{enumi}{11}
\item
{\it Machine Learning}.
(https://www.coursera.org/learn/machine-learning/).
Courses: Linear and Logistic Regression, Regularization, Classification, Neural Networks, SVM.
Taught by Andrew Ng.
Stanford University.
Audited in August 2016.

\item
{\it Python for Everybody Specialization}.
(https://www.coursera.org/specializations/python/).
Courses: Programming for Everybody (Getting Started with Python), Python Data Structures, Using Python to Access Web Data, Using Databases with Python.
Taught by Charles Russell Severance.
University of Michigan.
Audited in July 2016.

\item
{\it Internet History, Technology, and Security}.
(https://www.coursera.org/learn/internet-history/).
History of internet and programming languages.
Taught by Charles Russell Severance.
University of Michigan.
Attended in July 2016.
\end{enumerate}

\newpage

%%%%%%%%%%%%%%%%%%%%%%%%%%%%%%%%%%%%%%%%%%%%%%%%%%%%%%%%%%%%%%%
%%%%%%%%%%%    PROJECTS SUPERVISED     %%%%%%%%%%%%%%%%%%%%%%%%
%%%%%%%%%%%%%%%%%%%%%%%%%%%%%%%%%%%%%%%%%%%%%%%%%%%%%%%%%%%%%%%

\section*{Projects Supervised}

\begin{enumerate}
\item
{\it T\'ecnicas de regresi\'on y an\'alisis de datos}.
Feb.~2018-Apr.~2018 (135 h). Programa de Pr\'acticas Externas Curriculares (Grado de Ciencias F{\'\i}sicas, UAM).
Student: Ricardo Olivas Gonz\'alez.
%(17-0013-DEV-01).
%12 febrero 2018 - 12 abril 2018.

\item
{\it Optimizaci\'on de potencia fotovoltaica instalada para un sistema de autoconsumo bajo la regulaci\'on del RD 900/2015}.
Feb.~2016-May~2016 (250 h). Proyecto del M\'aster en Energ{\'\i}as y Combustibles para el Futuro (UAM).
Student: Enrique Iborra Pernichi.
%(15 0020 DEV 01).
%22 febrero 2016 - 20 mayo 2016.

\item
{\it Medida de coeficientes de correcci\'on de curvas I-V para medidas en condiciones de campo} (conti\-nu\-aci\'on).
Nov.~2015-Feb.~2016 (135 h). Programa de Pr\'acticas Externas Curriculares (Grado de Ciencias F{\'\i}sicas, UAM).
Student: Daniel Vidal Ortiz.
%(14-0017-DEV continuaci\'on).
%16 noviembre 2015 - 25 febrero 2016.

\item
{\it Medida de coeficientes de correcci\'on de curvas I-V para medidas en condiciones de campo}.
May~2015-Jul.~2015 (135 h). Programa de Pr\'acticas Externas Curriculares (Grado de Ciencias F{\'\i}sicas, UAM).
Student: Mariano Domingo Jim\'enez S\'anchez.
%(14-0017-DEV).
%26 mayo 2015 - 16 julio 2015.

\item
{\it Propuesta de un esquema de monitorizaci\'on en tiempo real del sistema de m\'odulos fotovoltaicos de la empresa Yingli Green Energy Spain}.
Feb.~2015-May~2015 (250 h). Proyecto del M\'aster en Energ{\'\i}as y Combustibles para el Futuro (UAM).
Student: Jos\'e Alberto Florez.
%(13-0037-DEV continuaci\'on).
%23 febrero 2015 - 27 mayo 2015.

\item
{\it Desarrollo del sistema de adquisici\'on y an\'alisis de datos del sistema fotovoltaico instalado en Yingli}.
Feb.~2014-Apr.~2014 (250 h). Proyecto del M\'aster en Energ{\'\i}as y Combustibles para el Futuro (UAM).
Student: Juan Ram\'on Diego Cagigas.
%(13-0037-DEV).
%3 febrero 2014 - 11 abril 2014.

\item
{\it Electroluminescence defects analysis and classification}.
Aug.~2013-Sept.~2013 (120 h). Programa de Pr\'acticas Externas Curriculares (Grado de Ciencias F{\'\i}sicas, UAM).
Student: Aitor Balda Jurado.
%(13-0008-DEV).
%1 agosto 2013 - 10 septiembre 2013.

\item
{\it Puesta en marcha de trazador de curvas I-V Daystar DS-100C}.
Sept.~2012-Oct.~2012 (120 h). Programa de Pr\'acticas Externas Curriculares (Grado de Ciencias F{\'\i}sicas, UAM).
Student: Mois\'es O\~noro Sala�ces. 
%(12-0032-DEV)
%6 septiembre 2012 - 25 octubre 2012.
\end{enumerate}

%%%%%%%%%%%%%%%%%%%%%%%%%%%%%%%%%%%%%%%%%%%%%%%%%%%%%%%%%%%%%%%
%%%%%%%%%%%    OTHER MERITS     %%%%%%%%%%%%%%%%%%%%%%%%%%%%%%%
%%%%%%%%%%%%%%%%%%%%%%%%%%%%%%%%%%%%%%%%%%%%%%%%%%%%%%%%%%%%%%%

%\section*{Other Merits}

\hrule 

\bigskip

\noindent
\version




\end{document}
