% jmroman_cv-es.tex, latex file
\newcommand{\version}{Madrid, 5 Marzo 2018.}

\documentclass{article}
\usepackage{multirow}
\textwidth 6.5in  %165mm
%\textheight 8.9in %226mm-letter
\textheight 9.5in %241mm-a4paper
\topmargin -0.4in %10mm
\oddsidemargin 0pt
\evensidemargin 0pt
\parindent 0pt

\pagestyle{empty}


\begin{document}
%%%%%%%%%%%%%%%%%%%%%%%%%%%%%%%%%%%%%%%%%%%%%%%%%%%%%%%%%%%%%%
%%%%%%%%%%%     PERSONAL DATA    %%%%%%%%%%%%%%%%%%%%%%%%%%%%%
%%%%%%%%%%%%%%%%%%%%%%%%%%%%%%%%%%%%%%%%%%%%%%%%%%%%%%%%%%%%%%

{\huge \bf  Jos\'e Mar\'{\i}a Rom\'an Fa\'undez}

\bigskip \medskip

\begin{tabular}{lrl}
{\bf Yingli Green Energy Europe} & & {\medskip} \\
{\small Pol. Ind. Sur - Ctra. N-I km 32,1}
   & {\small {\bf Tel:}} & {\small +34\,91\,843\,6726} \\
{\small E-28750 San Agust\'{\i}n de Guadalix (Madrid)}
   & {\small {\bf Fax:}} & {\small +34\,91\,848\,7915} \\
{\small Espa\~na}
   & {\small {\bf e-mail:}} & {\small jm.roman@yinglisolar.com} \\
{\small }
   & {\small {\bf webpage:}} & {\small http://www.yinglisolar.com/}
%{\small C/ Gudari 20, 4 dcha-C}
%   & {\small {\bf Tel:}} & {\small +34\,637\,757\,325} \\
%   & {\small {\bf e-mail:}} & {\small josemaria.roman@uam.es} \\
%{\small E-48340 AMOREBIETA (Vizcaya) Espa\~na}
%   & {\small {\bf e-mail:}} & {\small josemaria.roman@uam.es} \\
%{\small ~}
%   & {\small {\bf webpage:}} & {\small http://www.uam.es/josemaria.roman/}
\end{tabular}

\bigskip

\hrule

%%%%%%%%%%%%%%%%%%%%%%%%%%%%%%%%%%%%%%%%%%%%%%%%%%%%%%%%%%%%%%
%\begin{tabular}{lrl}
%\multirow{4}{102mm}{\huge \bf  Jos\'{e} Mar\'{\i}a Rom\'{a}n Fa\'{u}ndez}
%& {\small {\bf Fecha de nacimiento:}} & {\small 18 de Marzo de 1971} \\
%& {\small {\bf Lugar de nacimiento:}} & {\small Zamora} \\
%& {\small {\bf Nacionalidad:}} & {\small Espa\~nola} \\
%& {\small {\bf D.N.I.:}} & {\small 11\,894\,268\,-\,W} 
%\end{tabular}

%\begin{tabular}{lrl}
%{\small C/ Gudari 20, 4 dcha.-C}
%   & {\small {\bf Tel:}} & {\small 94\,673\,3814} \\
%{\small 48340 AMOREBIETA}
%   & {\small {\bf e-mail:}} & {\small jmroman@europe.com} \\
%{\small (Vizcaya)}
%   & {\small {\bf webpage:}} & {\small http://w3.physics.uiuc.edu/$\sim$romanfau/}
%\end{tabular}

%\bigskip

%\hrule

%%%%%%%%%%%%%%%%%%%%%%%%%%%%%%%%%%%%%%%%%%%%%%%%%%%%%%%%%%%%%%
%%%%%%%%%%%%    SUMMARY     %%%%%%%%%%%%%%%%%%%%%%%%%%%%%%%%%%
%%%%%%%%%%%%%%%%%%%%%%%%%%%%%%%%%%%%%%%%%%%%%%%%%%%%%%%%%%%%%%

\section*{Resumen}

Director del Laboratorio y Servicios O{\&}M en el R{\&}D, Aftersales Service Center de Yingli Green Energy Europe.  Investigador de Proyecto en el Centro L\'aser UPM. Director de Laboratorio de Calidad y Control de Sistemas Fotovoltaicos durante cinco a\~nos en INGENIA Solar Energy y CENER. Technical Manager de l\'{\i}nea de producci\'on de m\'odulos vidrio-vidrio para integraci\'on arquitect\'onica. Implantaci\'on de sistemas de gesti\'on de la calidad ISO 17025 e ISO 9001. Doctor en Ciencias F\'{\i}sicas con dos a\~nos de experiencia investigadora en EE.UU. Publicaciones en revistas internacionales. Cinco a\~nos en el extranjero. Ingl\'es fluido.

\bigskip

\hrule

%%%%%%%%%%%%%%%%%%%%%%%%%%%%%%%%%%%%%%%%%%%%%%%%%%%%%%%%%%%%%%
%%%%%%%%%%%%    CURRENT SITUATION     %%%%%%%%%%%%%%%%%%%%%%%%
%%%%%%%%%%%%%%%%%%%%%%%%%%%%%%%%%%%%%%%%%%%%%%%%%%%%%%%%%%%%%%

\section*{Situaci\'on Actual}

{\bf Director del Laboratorio y Servicios O{\&}M en el R{\&}D, Aftersales Service Center}

\medskip
Feb.~2012-Presente.
{\bf Yingli Green Energy Europe},
Madrid, Espa\~na.

\begin{itemize}\itemsep 0pt
\item Supervisi\'on de Servicios de O{\&}M, coordinaci\'on con contratistas y clientes.
\item Cursos de formaci\'on de ensayos de Operaci\'on y Mantenimiento de plantas FV.
\item Monitorizaci\'on del sistema BIPV integrado en el edificio de YGEE. 
\item Instalaci\'on de un sistema FV aislado y planificaci\'on de proyecto de gesti\'on de la demanda energ\'etica.
\item Gesti\'on de proyectos de postventa e I+D.
\item An\'alisis y ensayos de producto en laboratorio y en campo.
\item Definici\'on de nuevos ensayos y conformidad con las directrices de ISO 17025.
\item Miembro del SC-82 de AENOR para la normalizaci\'on de m\'odulos y sistemas FV.
\item Due-diligence t\'ecnicas de plantas FV.
\item Auditor\'{\i}as de plantas de fabricaci\'on de m\'odulos FV.
\end{itemize}

%%%%%%%%%%%%%%%%%%%%%%%%%%%%%%%%%%%%%%%%%%%%%%%%%%%%%%%%%%%%%%
%%%%%%%%%%%%    PROFESSIONAL EXPERIENCE     %%%%%%%%%%%%%%%%%%
%%%%%%%%%%%%%%%%%%%%%%%%%%%%%%%%%%%%%%%%%%%%%%%%%%%%%%%%%%%%%%

\section*{Experiencia Profesional}

{\bf Investigador de Proyecto}

\medskip
Feb.~2011-Ene.~2012.
{\bf Centro L\'aser UPM},
Madrid, Espa\~na.

\begin{itemize}\itemsep 0pt
\item Investig\'o sobre la formaci\'on de contactos met\'alicos para c\'elulas solares mediante l\'aser.
\item Operaci\'on de l\'aseres q-switched y continuos, microscopio Confocal, SEM y espectr\'oscopo EDX.
\end{itemize}

{\bf Director del Laboratorio de Calidad y Control de Sistemas Fotovoltaicos}

\medskip
Mayo~2007-Ene.~2011.
{\bf INGENIA Solar Energy},
Albacete, Espa\~na.

\begin{itemize}\itemsep 0pt
\item Medidas de control de calidad de m\'odulos fotovoltaicos bajo normas internacionales.
\item Inspecciones y medidas de rendimiento de plantas FV.
\item Dise\~no y dimensionamiento de plantas FV y c\'alculos de producci\'on.
\item Implantaci\'on de sistema de gesti\'on de calidad de laboratorios ISO 17025.
\end{itemize}

{\bf Jefe del Servicio de Sistemas Fotovoltaicos}

\medskip
Abr.~2006-Abr.~2007. 
{\bf CENER - Centro Nacional de Energ\'{\i}as Renovables},
Navarra, Espa\~na.

\begin{itemize}\itemsep 0pt
\item Jefe del Laboratorio de Ensayo de M\'odulos Fotovoltaicos.
\item Jefe del grupo de Dise\~no de Instalaciones Fotovoltaicas.
\item Implantaci\'on de sistema de gesti\'on del Servicio.
\item Gesti\'on de proyectos comerciales y de desarrollo.
\end{itemize}

%\newpage

%%%%%%%%%%%%%%%%%%%%%%%%%%%%%%%%%%%%%%%%%%%%%%%%%%%%%%%%%%%%%%
{\bf Technical Manager de la l\'{\i}nea de producci\'on de m\'odulos fotovoltaicos}

\medskip
Oct.~2004-Mar.~2006. 
{\bf Romag Ltd.} 
Consett, Co. Durham, UK.

\begin{itemize}\itemsep 0pt
\item Dise\~no del proceso de laminaci\'on para m\'odulos solares vidrio-vidrio a medida, y est\'andar (vidrio-Tedlar).
\item Definici\'on del proceso de producci\'on para m\'odulos solares vidrio-vidrio a medida, as\'{\i} como para m\'odulos est\'andar vidrio-Tedlar para la calificaci\'on bajo la norma IEC 61215.
\item Adaptaci\'on del proceso de producci\'on a la norma de gesti\'on de calidad ISO 9001.
\item Formaci\'on de personal en el proceso de producci\'on y en la utilizaci\'on de equipos de laminaci\'on de vac\'{\i}o y medidores de flash.
\item Trato con clientes y proveedores.
\end{itemize}

%%%%%%%%%%%%%%%%%%%%%%%%%%%%%%%%%%%%%%%%%%%%%%%%%%%%%%%%%%%%%%
%%%%%%%%%%%%%   EDUCATION   %%%%%%%%%%%%%%%%%%%%%%%%%%%%%%%%%%
%%%%%%%%%%%%%%%%%%%%%%%%%%%%%%%%%%%%%%%%%%%%%%%%%%%%%%%%%%%%%%

\section*{Titulaci\'{o}n Acad\'{e}mica}

{\bf Curso de Alta Direcci\'on Empresarial (CADE)}

\medskip
Nov.~2009-Jun.~2010. {\bf FEDA-Fundesem}. Albacete, Espa\~na.

\begin{itemize}\itemsep 0pt
\item {\bf Objetivo:} Desarrollar conocimientos y habilidades b\'asicos para la gesti\'on global de la empresa.
\begin{itemize} 
\item Direcci\'on estrat\'egica:  Direcci\'on estrat\'egica y Entorno econ\'omico.
\item Marketing y comercializaci\'on:  Planificaci\'on de Marketing y Planificaci\'on de Ventas.
\item Econ\'omica y Financiera:  An\'alisis financiero y Control de gesti\'on.
\item Recursos Humanos y Habilidades Directivas:  Comunicaci\'on y Direcci\'on de personas.
\end{itemize}
\end{itemize}

%%%%%%%%%%%%%%%%%%%%%%%%%%%%%%%%%%%%%%%%%%%%%%%%%%%%%%%%%%%%%%
\medskip
{\bf M\'aster Europeo en Energ\'{\i}as Renovables de la Agencia EUREC}

\medskip
Oct.~2003-Sept.~2004. {\bf Universidad de Zaragoza}. Zaragoza, Espa\~na.

\begin{itemize}\itemsep 0pt
\item {\bf Introducci\'on:} CIRCE, Universidad de Zaragoza, Espa\~na.
\begin{itemize} 
\item Introducci\'on a los aspectos t\'ecnicos y socio-econ\'omicos de las Energ\'{\i}as Renovables: E\'olica, Solar Fotovoltaica, Solar T\'ermica, Hidroel\'ectrica y Biomasa.
\end{itemize}
\item {\bf Especializaci\'on:} Energ\'{\i}a Solar {\bf Fotovoltaica}, Northumbria University, Newcastle, UK.
\begin{itemize}
\item F\'{\i}sica, dise\~no y tecnolog\'{\i}a de c\'elulas solares y m\'odulos y dise\~no de sistemas fotovoltaicos.
\end{itemize}
\item {\bf Proyecto Final:} Architectural PV solar modules production line: Process description and implementation.
\begin{itemize}
\item Jun.~2004-Sept.~2004. Romag Ltd., Consett, Co. Durham, UK.
\item Dise\~no del proceso de laminaci\'on para m\'odulos solares vidrio-vidrio a medida.
\end{itemize}
\end{itemize}

%%%%%%%%%%%%%%%%%%%%%%%%%%%%%%%%%%%%%%%%%%%%%%%%%%%%%%%%%%%%%%
\newpage
\medskip
{\bf Doctor en Ciencias F\'{\i}sicas}

\medskip
Oct.~1994-Oct.~1998. {\bf Universitat de Barcelona}. Barcelona, Espa\~na.

\begin{itemize}\itemsep 0pt
\item Formaci\'on en Teor\'{\i}a Cu\'{a}ntica de Campos, Relatividad General, F\'{\i}sica Nuclear, Ecuaciones estoc\'{a}ticas y ruido, y Teor\'{\i}a de Cuerdas.
\item {\bf Tesis:} Low Energy Properties of Magnetic Systems.
\begin{itemize}
\item {\bf Director:} Joan Soto Riera. 
\item Combin\'{o} cristalograf\'{\i}a, \'{o}ptica y teor\'{\i}as efectivas 
de campos para determinar la din\'{a}mica de las excitaciones de ondas de 
spin y efectos no rec\'{\i}procos en ferromagnetos y antiferromagnetos.
\item Aplic\'{o} una t\'{e}cnica num\'{e}rica recursiva al c\'{a}lculo 
de propiedades del estado fundamental de escaleras de spin cu\'{a}ntico.
\end{itemize}
\end{itemize}

%%%%%%%%%%%%%%%%%%%%%%%%%%%%%%%%%%%%%%%%%%%%%%%%%%%%%%%%%%%%%%
\medskip
{\bf Licenciado en Ciencias F\'{\i}sicas, especialidad en F\'{\i}sica del Estado S\'{o}lido}
%Calificaci\'{o}n: 8.6/10.0.

\medskip
Sept.~1989-Jun.~1994. 
{\bf Univ.\ del Pa\'{\i}s Vasco - Euskal Herriko Unibertsitatea}. Leioa, Espa\~na.

\begin{itemize}\itemsep 0pt
\item Formaci\'on en Cristalograf\'{\i}a, Din\'{a}mica de red y fonones, 
Estructura electr\'{o}nica y semiconductores, Mec\'{a}nica Cu\'{a}ntica, \'Optica, Materiales Diel\'{e}ctricos, Teor\'{\i}a de Grupos y F\'{\i}sica de Part\'{\i}culas, junto con varios conjuntos de experimentos relacionados.
\end{itemize}

%%%%%%%%%%%%%%%%%%%%%%%%%%%%%%%%%%%%%%%%%%%%%%%%%%%%%%%%%%%%%%
%%%%%%%%%%     RESEARCH EXPERIENCE     %%%%%%%%%%%%%%%%%%%%%%%
%%%%%%%%%%%%%%%%%%%%%%%%%%%%%%%%%%%%%%%%%%%%%%%%%%%%%%%%%%%%%%

\section*{Experiencia Investigadora}

Jul.~2001-Nov.~2003. 
{\bf Instituto de F\'{\i}sica Te\'orica, CSIC-UAM}. 
Madrid, Espa\~na. 

\begin{itemize}\itemsep 0pt
\item Estudi\'o el flujo c\'{\i}clico del RG y sus efectos en 
superconductores y cadenas de spin.

\item Obtuvo las principales caracter\'{\i}sticas del estado fundamental 
y las excitaciones en granos peque\~nos de materiales con correlaciones 
superconductoras.
\end{itemize}

%\pagebreak

%%%%%%%%%%%%%%%%%%%%%%%%%%%%%%%%%%%%%%%%%%%%%%%%%%%%%%%%%%%%%%
Ene.~2001-Jul.~2001. {\bf Universidade de \'Evora}. \'Evora, Portugal.

\begin{itemize}\itemsep 0pt
\item Colabor\'o con el ISTAS en la preparaci\'on de un programa de 
conferencias cient\'{\i}ficas.

\item Describi\'{o} las excitaciones de energ\'{\i}a finita del modelo 
de Hubbard 1-D como combinaciones de spinones y holones.
\end{itemize}

%%%%%%%%%%%%%%%%%%%%%%%%%%%%%%%%%%%%%%%%%%%%%%%%%%%%%%%%%%%%%%
Nov.~1998-Dic.~2000. {\bf University of Illinois at Urbana-Champaign}. 
Urbana, Illinois, EE.UU.

\begin{itemize}\itemsep 0pt
\item Estudi\'{o} la supresi\'{o}n del par\'{a}metro de orden superconductor 
alrededor de impurezas magn\'{e}ticas en superconductores de onda-$d$.

\item Extendi\'{o} la teor\'{\i}a efectiva de las ondas de spin 
a fases {\em canted}, y la aplic\'{o} al estudio de las excitaciones 
magn\'{e}ticas de las manganitas dopadas.
\end{itemize}

%%%%%%%%%%%%%%%%%%%%%%%%%%%%%%%%%%%%%%%%%%%%%%%%%%%%%%%%%%%%%%
Oct.~1994-Oct.~1998. {\bf Universitat de Barcelona}. Barcelona, Espa\~na.

\begin{itemize}\itemsep 0pt
\item Defini\'{o} un modelo continuo para obtener el diagrama de fases del 
estado fundamental de las manganitas dopadas como funci\'{o}n del dopaje.

\item Combin\'{o} cristalograf\'{\i}a, \'{o}ptica y teor\'{\i}as efectivas 
de campos para determinar la din\'{a}mica de las excitaciones de ondas de 
spin y efectos no rec\'{\i}procos en ferromagnetos y antiferromagnetos.
\end{itemize}

%%%%%%%%%%%%%%%%%%%%%%%%%%%%%%%%%%%%%%%%%%%%%%%%%%%%%%%%%%%%%%
Oct.~1997-Dic.~1997. 
{\bf Consejo Superior de Investigaciones Cient\'{\i}ficas (CSIC)}. Madrid, Espa\~na. 

\begin{itemize}\itemsep 0pt
\item Aplic\'{o} una t\'{e}cnica num\'{e}rica recursiva al c\'{a}lculo 
de propiedades del estado fundamental de escaleras de spin cu\'{a}ntico.
\end{itemize}

%\newpage

%%%%%%%%%%    TEACHING EXPERIENCE     %%%%%%%%%%%%%%%%%%%%%%%

\subsection*{Experiencia Docente}

\begin{itemize}\itemsep 0pt
\item 1998.
Imparti\'{o} clases de problemas de la asignatura de Octavo Semestre 
{\it F\'{\i}sica Nuclear i Parti\-cu\-les}. 
Facultat de F\'{\i}sica. 
Universitat de Barcelona. 
Barcelona, Espa\~na.
%30 hours.

\item 1995.
Imparti\'{o} clases de problemas de la asignatura de Segundo Semestre 
{\it Mec\'{a}nica i Ones}. 
Facultat de F\'{\i}sica. 
Universitat de Barcelona. 
Barcelona, Espa\~na.
%45 hours.
\end{itemize}

%%%%%%%%%%%    PUBLICATIONS     %%%%%%%%%%%%%%%%%%%%%%%%%%%%%%%

\subsection*{Publicaciones}

\begin{itemize}\itemsep 0pt
\item Catorce publicaciones en revistas internacionales con evaluadores externos.
\end{itemize}

%%%%%%%%%    SEMINARS AND COMMUNICATIONS    %%%%%%%%%%%%%%%%%%%%%%%

\subsection*{Seminarios y Comunicaciones en Congresos}

\begin{itemize}\itemsep 0pt
\item Trece presentaciones orales y dos posters invitados en varias 
universidades y conferencias internacionales.
\end{itemize}

%%%%%%%%%    COURSES AND CONFERENCES    %%%%%%%%%%%%%%%%%%%%%%%%%%

\subsection*{Cursos y Conferencias}

\begin{itemize}\itemsep 0pt
\item Asistencia a diez cursos y conferencias internacionales.
\end{itemize}

%\newpage

%%%%%%%%%%%%%%%%%%%%%%%%%%%%%%%%%%%%%%%%%%%%%%%%%%%%%%%%%%%%%%
%%%%%%%%%%%%%%    FELLOWSHIPS    %%%%%%%%%%%%%%%%%%%%%%%%%%%%%
%%%%%%%%%%%%%%%%%%%%%%%%%%%%%%%%%%%%%%%%%%%%%%%%%%%%%%%%%%%%%%

\section*{Becas}

\begin{itemize}\itemsep 0pt
\item Ene.~2001-Jul.~2001.
Beca postdoctoral de la Fundaci\'on para la Ciencia y la 
Tecnolog\'{\i}a del Gobierno Portugu\'es.

\item Nov.~1998-Sept.~2000. 
Beca postdoctoral FPI del Dpto.\ de Educaci\'{o}n, Universidades 
e Investigaci\'{o}n del Gobierno Vasco. 

\item Oct.~1994-Sept.~1998. 
Beca predoctoral FPI del Dpto.\ de Educaci\'{o}n, Universidades 
e Investigaci\'{o}n
\linebreak
del Gobierno Vasco.
\end{itemize}


%%%%%%%%%%%%%%%%%%%%%%%%%%%%%%%%%%%%%%%%%%%%%%%%%%%%%%%%%%%%%%
%%%%%%%%%%%%     PROFESSIONAL SKILLS      %%%%%%%%%%%%%%%%%%%%
%%%%%%%%%%%%%%%%%%%%%%%%%%%%%%%%%%%%%%%%%%%%%%%%%%%%%%%%%%%%%%

\section*{Aptitudes Profesionales}

\begin{itemize}\itemsep 0pt
\item Excelente organizaci\'on para la gesti\'on empresarial con ajuste a normativas de calidad ISO 9001 e ISO 17025.

\item Extensa experiencia en la resoluci\'{o}n de pro\-ble\-mas 
matem\'{a}ticos y f\'{\i}sicos complejos mediante la combinaci\'{o}n 
de t\'{e}cnicas anal\'{\i}ticas y num\'{e}ricas.

\item Experiencia en trabajo interdisciplinario con r\'{a}pida 
adaptaci\'{o}n a nuevos temas de trabajo. R\'{a}pida capacidad de aprendizaje y capacidad docente y divulgativa.
\end{itemize}

%%%%%%%%%%%%%%%%%%%%%%%%%%%%%%%%%%%%%%%%%%%%%%%%%%%%%%%%%%%%%%
%%%%%%%%%%%%%%%    COMPUTER SKILLS      %%%%%%%%%%%%%%%%%%%%%%
%%%%%%%%%%%%%%%%%%%%%%%%%%%%%%%%%%%%%%%%%%%%%%%%%%%%%%%%%%%%%%

\section*{Experiencia Computacional}

\begin{itemize}\itemsep 0pt
\item Experto en ofim\'atica y c\'alculo: MS Word, Excel y Access, \TeX y \LaTeX.

\item Conocimiento de Java, HTML (http://dftuz.unizar.es/), Auto-CAD.

\item Amplia experiencia en programaci\'{o}n en C++, FORTRAN, Python, Mathematica y Octave en la resoluci\'{o}n num\'{e}rica de pro\-ble\-mas.

\item Combin\'{o} c\'{o}digo C++, librer\'{\i}as FORTRAN y XmGrace para obtener salidas gr\'{a}ficas en tiempo real.

\item Aplicaci\'on de monitorizaci\'on de lectura y an\'alisis de datos con Python y MS SQLServer.

\item Administraci\'{o}n de Linux a nivel intermedio.  
Usuario de plataformas UNIX (AIX, IRIX), Windows y Windows NT.
\end{itemize}


%%%%%%%%%%%%%%%%%%%%%%%%%%%%%%%%%%%%%%%%%%%%%%%%%%%%%%%%%%%%%%
%%%%%%%%%%%%%%     LANGUAGES     %%%%%%%%%%%%%%%%%%%%%%%%%%%%%
%%%%%%%%%%%%%%%%%%%%%%%%%%%%%%%%%%%%%%%%%%%%%%%%%%%%%%%%%%%%%%

\section*{Idiomas}
\begin{itemize}\itemsep 0pt
\item Espa\~nol, Ingl\'{e}s fluido y conocimiento moderado de Portugu\'es, 
Catal\'{a}n y Euskera.
\end{itemize}

%\section*{Otros M\'{e}ritos}

%%%%%%%%%%%%%%%%%%%%%%%%%%%%%%%%%%%%%%%%%%%%%%%%%%%%%%%%%%%%%%%

\bigskip

\hrule

\newpage

%%%%%%%%%%%%%%%%%%%%%%%%%%%%%%%%%%%%%%%%%%%%%%%%%%%%%%%%%%%%%%%
%%%%%%%%%%%    PUBLICATIONS     %%%%%%%%%%%%%%%%%%%%%%%%%%%%%%%
%%%%%%%%%%%%%%%%%%%%%%%%%%%%%%%%%%%%%%%%%%%%%%%%%%%%%%%%%%%%%%%

\section*{Publicaciones}

\begin{enumerate}
\item A. LeClair, J. M. Rom\'an and G. Sierra,
{\it Log-periodic Behavior of Finite-size Effects in Field Theories with RG Limit Cycles}.
{\it Nucl.~Phys.} {\bf B700} (2004) 407-435.
%(hep-th/0312141).
%November 15, 2004.

\item G. Sierra, J. M. Rom\'an and J. Dukelsky, 
{\it The Elementary Excitations of the BCS Model in the Canonical Ensemble}.
{\it Int.~J.~Mod.~Phys.} {\bf A19S2} (2004) 381-395.
%(cond-mat/0301417).
%May, 2004.

\item J. M. P. Carmelo, J. M. Rom\'{a}n and K.Penc,
{\it Charge and Spin Quantum Fluids Generated by Many-Electron Interactions}.
{\it Nucl.~Phys.} {\bf B683} (2004) 387-422.
%(cond-mat/0302044).
%March 12, 2004.

\item A. LeClair, J. M. Rom\'{a}n and G. Sierra, 
{\it Russian Doll Renormalization Group and Superconductivity}. 
{\it Phys.~Rev.} {\bf B69} (2004) 020505 (4 pages).
%(cond-mat/0211338).
%January 27, 2004.

\item A. LeClair, J. M. Rom\'{a}n and G. Sierra, 
{\it Russian Doll Renormalization Group, Kosterlitz-Thouless Flows, 
and the Cyclic sine-Gordon Model}. 
{\it Nucl.~Phys.} {\bf B675} (2003) 584-606.
%(hep-th/0301042).
%December 29, 2003.

\item J. Dukelsky, J. M. Rom\'an and G. Sierra,
Comment on {\it Polynomial-time Simulation of Pairing Models 
on a Quantum Computer}.
{\it Phys.~Rev.~Lett.} {\bf 90} (2003) 249803.
%(quant-ph/0305139).
%June 20, 2003

\item J. M. Rom\'an, G. Sierra and J. Dukelsky, 
{\it Elementary Excitations of the BCS Model in the Canonical Ensemble},
{\it Phys.~Rev.} {\bf B67} (2003) 064510 (6 pages).
%(cond-mat/0207640).
%February 28, 2003.

\item J. M. Rom\'an, G. Sierra and J. Dukelsky, 
{\it Large $N$ Limit of the Exactly Solvable BCS Model: 
Analytics versus Numerics},
{\it Nucl.~Phys.} {\bf B634} (2002) 483-510.
%(cond-mat/0202070).
%July 15, 2002.

\item J. M. Rom\'an and J. Soto, 
{\it Spin Waves in Canted Phases: An Application to Doped Manganites}, 
{\it Phys.~Rev.} {\bf B62} (2000) 3300-3315.
%(cond-mat/9911471).
%August 1, 2000.

\item J. M. Rom\'an and J. Soto, 
{\it Continuum Double Exchange Model},
{\it Phys.~Rev.} {\bf B59} (1999) 11418-11423.
%(cond-mat/9810389).
%May 1, 1999.

\item J. M. Rom\'an and J. Soto, 
{\it Spin Wave Mediated Non-Reciprocal Effects in Antiferromagnets}, 
{\it Ann.~Phys.} {\bf 273} (1999) 37-57.
%(cond-mat/9709299).
%April 10, 1999.

\item J. M. Rom\'an and J. Soto, 
{\it Effective Field Theory Approach to Ferromagnets and Antiferromagnets 
in Crystalline Solids}, 
{\it Int.~J.~Mod.~Phys.} {\bf B13} (1999) 755-789.
%(cond-mat/9709298).
%March 20, 1999.

\item J. M. Rom\'an, G. Sierra, J. Dukelsky and M. A. Mart\'{\i}n-Delgado, 
{\it The Matrix Product Approach to Quantum Spin Ladders}, 
{\it J.~Phys.} {\bf A31} (1998) 9729-9759.
%(cond-mat/9802150).
%December 4, 1998.

\item J. M. Rom\'an and R. Tarrach, 
{\it The Regulated Four-Parameter One-Dimensional Point Interaction}, 
{\it J.~Phys.} {\bf A29} (1996) 6073-6085.
%(hep-th/9511010).
%September 21, 1996.
\end{enumerate}

%\section*{Preprints}

%%%%%%%%%%%%%%%%%%%%%%%%%%%%%%%%%%%%%%%%%%%%%%%%%%%%%%%%%%%%%%
%%%%%%%%%%%%%%%%%%%%    SEMINARS    %%%%%%%%%%%%%%%%%%%%%%%%%%
%%%%%%%%%%%%%%%%%%%%%%%%%%%%%%%%%%%%%%%%%%%%%%%%%%%%%%%%%%%%%%

\section*{Seminarios}

\begin{enumerate}
\item
{\it Cyclic RG Theories and c-Function Behavior in Finite Size Systems}.
Dpt.~d'Estruc\-tura i Constituents de la Mat\`eria.
Facultat de F\'{\i}sica. Universitat de Barcelona.
Barcelona, Spain.
Jan.~8, 2004.

\item
{\it Limit Cycles in the Renormalization Group of the BCS and the sine-Gordon models}. 
Dpto.\ de F\'{\i}sica Te\'orica. Universidad de Zaragoza.
Zaragoza, Spain.
Oct.~29, 2003.

\item
{\it Russian Doll Renormalization Group and Superconductivity}. 
Instituto de Ciencias de Materiales de Madrid, CSIC. 
Madrid, Spain.
Mar.~20, 2003.

\item
{\it Superconductivity in Small Grains: Ground State Thermodynamic 
Limit and Excitations}.
Dpto.~de F\'{\i}sica Aplicada. Universidad de Alicante.
Alicante, Spain.
Nov.~21, 2002.

\item
{\it Superconductivity in Small Grains: Ground State Thermodynamic 
Limit and Excitations}.
Dpt.~d'Estruc\-tura i Constituents de la Mat\`eria.
Facultat de F\'{\i}sica. Universitat de Barcelona.
Barcelona, Spain.
Jul.~4, 2002.

\item 
{\it Disentangling Canted Phases and Phase Separation Regions 
in Doped Manganites with Spin Waves}.
Centro de F\'{\i}sica das Interac\c c\~oes Fundamentais. 
Instituto Superior T\'ecnico.
Lisbon, Portugal.
Apr.~3, 2001.

\item 
{\it Disentangling Canted Phases and Phase Separation Regions 
in Doped Manganites with Spin Waves}.
Dpto. de F\'{\i}sica. 
Faculdade de Ci\^encias.
Universidade do Porto.
Oporto, Portugal.
Jan.~26, 2001.

\item 
{\it Disentangling Canted Phases and Phase Separation Regions 
in Doped Manganites with Spin Waves}.
Instituto de Ciencias de Materiales de Madrid, CSIC. 
Madrid, Spain.
Jan.~10, 2000.

\item 
{\it Doped Manganites Phase Diagram from a Continuum Double Exchange Model}. 
Dpto.~de F\'{\i}sica del Estado S\'olido. Facultad de Qu{\'\i}mica.  
Universidad del Pa\'{\i}s Vasco - Euskal Herriko Unibertsitatea (UPV-EHU). 
San Sebastian-Donostia, Spain.
Sept.~9, 1999.

\item 
%{\it Teor\'{\i}as Efectivas Aplicadas a Cristales 
%Antiferromagn\'{e}ticos: Efectos no Rec\'{\i}procos Mediados 
%por Ondas de Spin}
{\it Effective Field Theory Approach to Spin Wave Mediated 
Non-Reciprocal Effects in Antiferromagnetic Crystals}. 
Instituto de Ciencias de Materiales de Madrid, CSIC. 
Madrid, Spain.
Dec.~10, 1997. 

\item 
{\it Effective Field Theory Approach to Spin Wave Mediated 
Non-Reciprocal Effects in Antiferromagnetic Crystals}. 
Dpto.\ de F\'{\i}sica Te\'orica e Historia de la Ciencia. 
Facultad de Ciencias. 
Universidad del Pa\'{\i}s Vasco - Euskal Herriko Unibertsitatea (UPV-EHU). 
Leioa, Spain.
Nov.~13, 1997. 
\end{enumerate}

%%%%%%%%%%%%%%%%%%%%%%%%%%%%%%%%%%%%%%%%%%%%%%%%%%%%%%%%%%%%%%
%%%%%%%%%%%%%   COMMUNICATIONS IN CONGRESSES    %%%%%%%%%%%%%%
%%%%%%%%%%%%%%%%%%%%%%%%%%%%%%%%%%%%%%%%%%%%%%%%%%%%%%%%%%%%%%

\section*{Comunicaciones en Congresos}

\begin{enumerate}
\setcounter{enumi}{11}
\item
{\it Russian Doll Renormalization Group and Superconductivity}. 
Trobada de Nadal 2002 (Christmas Meeting 2002) del dpt.~d'Estructura 
i Constituents de la Mat\`eria. 
Barcelona, Spain.
Dec.~19-20, 2002.

\item
{\it Disentangling Canted Phases from Phase Separation Regions in Doped 
Manganites with Spin Waves}. 
Euroconference on {\sl Transport and Dynamics in Complex Electronic Materials}.
Oporto, Portugal. 
Sept.~3-7, 2001. 

\item 
{\it Disentangling Canted Phases from Phase
Separation Regions in Doped Manganites with Spin Waves}.
2000 March Meeting of the American Physical Society. 
Minneapolis, MN, USA. 
Mar.~20-24, 2000.

\item 
{\it Doped Manganites Phase Diagram from a Continuum Double Exchange Model}. 
Summer School on {\sl Exotic States in Quantum Nanostructures}. 
Windsor, UK. 
Aug.~16-29, 1999.
\end{enumerate}


%%%%%%%%%%%%%%%%%%%%%%%%%%%%%%%%%%%%%%%%%%%%%%%%%%%%%%%%%%%%%%
%%%%%%%%%%%%%%%%      COURSES AND CONFERENCES      %%%%%%%%%%%
%%%%%%%%%%%%%%%%%%%%%%%%%%%%%%%%%%%%%%%%%%%%%%%%%%%%%%%%%%%%%%

\section*{Cursos y Conferencias}

\begin{enumerate}
\item
Trobada de Nadal 2002 (Christmas Meeting 2002) del dpt.~d'Estructura 
i Constituents de la Mat\`eria. 
Universitat de Barcelona.
Barcelona, Spain.
Dec.~19-20, 2002.

\item
Euroconference on 
{\sl Transport and Dynamics in Complex Electronic Materials}.
Oporto, Portugal. 
Sept.~3-7, 2001. 

\item 
2000 March Meeting of the American Physical Society. 
Minneapolis, MN, USA. 
Mar.~20-24, 2000.

\item 
Summer School on 
{\it Exotic States in Quantum Nanostructures}.
Windsor, UK. 
Aug.~16-29, 1999.

\item 
Summer School on 
{\it Dynamic Correlations in Many Fermion Systems}. 
Vila Nova de Cerveira, Portugal. 
Jul.~14-25, 1997.

\item
Advanced School on {\it Non-Perturbative Quantum Field Theory}. 
Pe\~n\'{\i}scola, Spain. 
Jun.~2-6, 1997.

\item 
%{\it Jornadas sobre Teor\'{\i}a Cu\'antica de Campos en 
%Sistemas de Baja Dimensi\'on y Materia Condensada}. 
Conference on 
{\it Quantum Field Theory in Low Dimensional and Condensed Matter Systems}.
Instituto de Ciencias de Materiales, CSIC. Madrid, Spain. 
Nov.~7-8, 1996.

\item  
%IV Escuela de Oto\~no de F\'{\i}sica Te\'orica,
%{\it M\'etodos no Perturbativos en Teor\'{\i}a Cu\'antica de Campos}.
IV Autumn School of Theoretical Physics on
{\it Non-perturbative Methods in Quantum Field Theory}.
Santiago de Compostela, Spain. 
Sept.~2-14, 1996.

\item  
%Cursos de Verano de la Universidad Complutense, 
%{\it Strongly Correlated Magnetic and Superconducting Systems}.
Summer School of the Complutense University of Madrid on
{\it Strongly Correlated Magnetic and Superconducting Systems}.
San Lorenzo del Escorial, Spain. 
Jul.~15-19, 1996.

\item  
%III Escuela de Oto\~no de F\'{\i}sica Te\'orica, 
%{\it Introducci\'on al Modelo Standard de las Interacciones Fundamentales}. 
III Autumn School of Theoretical Physics on
{\it Introduction to the Standard Model of the Fundamental Interactions}.
Santiago de Compostela, Spain. 
Sept.~4-16, 1995. 
\end{enumerate}

%%%%%%%%%%%%%%%%%%%%%%%%%%%%%%%%%%%%%%%%%%%%%%%%%%%%%%%%%%%%%%

\hrule 

\bigskip

\noindent
\version

\end{document}
